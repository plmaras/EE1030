\let\negmedspace\undefined
\let\negthickspace\undefined
\documentclass[journal,12pt,twocolumn]{IEEEtran}
\usepackage{cite}
\usepackage{amsmath,amssymb,amsfonts,amsthm}
\usepackage{algorithmic}
\usepackage{graphicx}
\usepackage{textcomp}
\usepackage{xcolor}
\usepackage{txfonts}
\usepackage{enumitem}
\usepackage{mathtools}
\usepackage{gensymb}
\usepackage{comment}
\usepackage[breaklinks=true]{hyperref}
\usepackage{tkz-euclide} 
\usepackage{listings}
\usepackage{gvv}                                        
%\def\inputGnumericTable{}                                 
\usepackage[latin1]{inputenc}                                
\usepackage{color}                                            
\usepackage{array}                                            
\usepackage{longtable}                                       
\usepackage{calc}                                             
\usepackage{multirow}                                         
\usepackage{hhline}                                           
\usepackage{ifthen}                                           
\usepackage{lscape}
\usepackage{tabularx}
\usepackage{array}
\usepackage{float}


\newtheorem{theorem}{Theorem}[section]
\newtheorem{problem}{Problem}
\newtheorem{proposition}{Proposition}[section]
\newtheorem{lemma}{Lemma}[section]
\newtheorem{corollary}[theorem]{Corollary}
\newtheorem{example}{Example}[section]
\newtheorem{definition}[problem]{Definition}
\newcommand{\BEQA}{\begin{eqnarray}}
\newcommand{\EEQA}{\end{eqnarray}}
\newcommand{\define}{\stackrel{\triangle}{=}}
\theoremstyle{remark}
\newtheorem{rem}{Remark}
\begin{document}

\bibliographystyle{IEEEtran}
\vspace{3cm}
\title{Mathematical Induction and Binomial theorem}
\author{Golla Shriram - AI24BTech11010}

\maketitle
\newpage
\bigskip

\renewcommand{\thefigure}{\theenumi}
\renewcommand{\thetable}{\theenumi}


\section{E - Subjective Problems}
                                                                           

\begin{enumerate}
             \item Given that  \hfill{(1979)}
		    \\ $C_1+2C_2x+3C_3x^2+.............+2nC_{2n}x^{2n-1}= 2n(1+x)^{2n-1}$where $C_r=\frac{\brak{2n}!}{r!\brak{2n-r}!}\\  r=0,1,2,.............,2n$\\
     Prove that\\	$C^2_1-2C_2^2+3C_3^2-.............-2nC_{2n}^2  = (-1)^nnC_n.$ \\

	      \item Prove that $7^{2n} + \brak{2^{3n-2}} \brak{3^{n-1}} $ is divisible by 25 for any natural number n \hfill{(1982-5M)}  \\

             \item If $\brak{1+x}^n= C_0 + C_1x +C_2x^2+......+C_nx^n$ then show that the sum of products of $ C_i$'s taken
		     two at a time, represented by $\displaystyle\sum_{0 \leq i<j \leq n}$ $\displaystyle\sum C_i C_j$ is equal to $2^{2n-1}$-$\frac{\brak{2n}!} {2\brak{n!}^2}$ \hfill\brak{1983-3M}\\

	     \item Use mathematical Induction to prove : If n is any odd positive integer , then $n\brak{n^2-1}$ is divisible by 24. \hfill\brak{1983-2M}\\

	    \item If $p$ be a natural number then prove that $p^{n+1}+\brak{p+1}^{2n-1}$ is divisible by $p^2+p+1$for every positive integer n. \hfill\brak{1984-4M} \\

            \item Given  $ s_n = 1 + q + q^2 +....+q^n:$\\
		    $S_n = 1 + \frac{q+1}{2}+\brak{\frac{q+1}{2}}^2+.....+\brak{\frac{q+1}{2}}^n,q\neq1$ Prove that\\
		    ${}^{n+1}C_1+{}^{n+1}C_2s_1+{}^{n+1}C_3s_2+.....+{}^{n+1}C_ns_n=2^nS_n$     \hfill\brak{1984-4M}\\
	    \item Use method of mathematical Induction\\$ 2.7^n +3.5^n-5$ is divisible by 24 for all $n>0$ \hfill\brak{1985-5M}\\

	    \item Prove by mathematical induction that -
		    $\frac{\brak{2n}!}{2^{2n}\brak{n!}^2}\leq \frac{1}{\brak{3n+1}^{\frac{1}{2}}}$  for all postive Integers n.\hfill \brak{1987-3M}

	    \item Let $R =\brak{5\sqrt{5}+11}^{2n}$ and $f = R -\sbrak{R}$, where \sbrak denotes the greatest integer function.Prove  that$ Rf =4^{2n+4}$  \hfill \brak{1988-5M}\\

	    \item Using mathematical induction,prove that
		    ${}^mC_0{}^nC_k +{}^mC_1{}^nC_{k-1}+................+{}^mC_k{}^nC_0 ={}^{m+k}C_k$\\\hfill \brak{1989-3M}\\
	    \item Prove that \hfill\brak{1989-5M}\\$C_0-2^2C_1+3^2C_2-............+\brak{-1}^n\brak{n+1}^2C_n =0$ ,$n>2$ , where $C_r={}^nC_r$\\

	    \item Prove that   $ \frac{n^7}{7}+\frac{n^5}{5}+\frac{2n^3}{3}-\frac{n}{105}$ is an integer for every positive integer n. \hfill\brak{1990-2M}\\


	    \item Using induction or otherwise , prove that for any non-negative integers m,n,r and k,\\$\displaystyle\sum_{r=0}^{k}(n-m)\frac{(r+m)!}{m!}= \frac{(r+k+1)!}{k!}[\frac{n}{r+1}-\frac{k}{r+2}]$ \hfill\brak{1991-4M}

	    \item If $\displaystyle\sum_{r=0}^{2n}a_r\brak{x-2}^r=\displaystyle\sum_{r=0}^{2n}b_r$$\brak{x-3}^r$ and $a_k =1$ for all k $\geq$ n then show that $b_n = {}^{2n+1}C_{n+1}$ \hfill \brak{1992-6M}

	    \item Let  $p\leq 3$ be an integer and $\alpha,\beta$ be  the roots of\\  $x^2-\brak{p+1}x+1=0 $ using mathematical induction show that $\alpha^n +\beta^n$\\ (i) is an integer and   
		    (ii) is not divisible by p  \hfill\brak{1992-6M}
		    
 \end{enumerate}
 \end{document}

