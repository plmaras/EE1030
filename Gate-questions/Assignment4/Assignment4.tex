\let\negmedspace\undefined
\let\negthickspace\undefined
\documentclass[journal,12pt,twocolumn]{IEEEtran}
\usepackage{cite}
\usepackage{amsmath,amssymb,amsfonts,amsthm}
\usepackage{algorithmic}
\usepackage{graphicx}
\usepackage{textcomp}
\usepackage{xcolor}
\usepackage{txfonts}
\usepackage{tikz}
\usepackage{circuitikz}
\usepackage{enumitem}
\usepackage{mathtools}
\usepackage{gensymb}
\usepackage{comment}
\usepackage[breaklinks=true]{hyperref}
\usepackage{tkz-euclide} 
\usepackage{listings}
\usepackage{gvv}                                        
%\def\inputGnumericTable{}                                 
\usepackage[latin1]{inputenc}                                
\usepackage{color}                                            
\usepackage{array}                                            
\usepackage{longtable}                                       
\usepackage{calc}                                             
\usepackage{multirow}                                         
\usepackage{hhline}                                           
\usepackage{ifthen}                                           
\usepackage{lscape}
\usepackage{tabularx}
\usepackage{array}
\usepackage{float}



\newtheorem{theorem}{Theorem}[section]
\newtheorem{problem}{Problem}
\newtheorem{proposition}{Proposition}[section]
\newtheorem{lemma}{Lemma}[section]
\newtheorem{corollary}[theorem]{Corollary}
\newtheorem{example}{Example}[section]
\newtheorem{definition}[problem]{Definition}
\newcommand{\BEQA}{\begin{eqnarray}}
\newcommand{\EEQA}{\end{eqnarray}}
\newcommand{\define}{\stackrel{\triangle}{=}}
\theoremstyle{remark}
\newtheorem{rem}{Remark}
\begin{document}

\bibliographystyle{IEEEtran}

\vspace{3cm}
\title{ 2010-CE-1-13 }
\author{Golla Shriram - AI24BTech11010}

\maketitle
%\newpage
%\bigskip

\renewcommand{\thefigure}{\theenumi}
\renewcommand{\thetable}{\theenumi}

\section{ Q.1-Q.25 carry one mark each. }
                                                                           
 \begin{enumerate}
		 
	 \item The {lim}$_{x \rightarrow 0}$ $\frac{sin{  \sbrak{\frac{2}{3} x}}}{x}$ \hfill{(2010-CE)}

\begin{enumerate}
	\item $\frac{2}{3}$
	\item 1
	\item $\frac{3}{2}$
	\item $\infty$

\end{enumerate}


\item Two coins are simultaneously tossed. The probability of two heads simultaneously appearing is \hfill{(2010-CE)}

		\begin{enumerate}
		\item $\frac{1}{8}$ 

	\item $\frac{1}{6} $

	\item $\frac{1}{4} $

	\item $\frac{1}{2}$ 
		\end{enumerate}

\item The order and degree of the differential equation $\frac{d^{3}y}{dx^3} + 4 \sqrt{ (\frac{dy}{dx})^{3} +y^2}  =  0$ \hfill{(2010-CE)}

	\begin{enumerate}
\item 3 and 2
\item 2 and 3
\item 3 and 3
\item 3 and 1
	\end{enumerate}


\item Two people weighing $W$ each are sitting on a plank of length $L$ floating on water at $\frac{L}{4}$ from either end. Neglecting the weight of the plank. The bending moment at the centre of the plank is \hfill{(2010-CE)}

	\begin{enumerate}
     \item $\frac{WL}{8}$
\item $\frac{WL}{16}$
\item $\frac{WL}{32}$
\item zero
	\end{enumerate}

\item For the truss shown in figure, the force in member QR is  \hfill{(2010-CE)}
\begin{figure}[!ht]


\centering
\resizebox{8 cm}{ 5 cm}{
\begin{circuitikz}

\tikzstyle{every node}=[font=\small]
\draw (4.5,10.75) to[short, -o] (2,10.75) ;
\draw (4.5,12.75) to[short, -o] (2,12.75) ;
\draw [->, >=Stealth] (4.5,12.75) -- (4.5,9.75);
\draw [->, >=Stealth] (3,10.5) -- (2.25,10.5);
\draw [->, >=Stealth] (3.75,10.5) -- (4.5,10.5);
\node [font=\normalsize] at (2,13.25) {Q};
\node [font=\normalsize] at (4.75,11) {S};
\node [font=\normalsize] at (4.75,12.75) {R};
\node [font=\normalsize] at (5,10) {P};
\node [font=\normalsize] at (2,10.25) {T};
\node [font=\normalsize] at (3.5,10.5) {L};
\node [font=\normalsize] at (5.5,11.75) {L};
\draw [short] (5.25,12.75) -- (5.75,12.75);
\draw [short] (5.25,10.75) -- (5.75,10.75);
\draw [->, >=Stealth] (5.5,12) -- (5.5,12.75);
\draw [->, >=Stealth] (5.5,11.5) .. controls (5.5,11.75) and (5.5,11.25) .. (5.5,10.75) ;
\draw [short] (2,10.75) -- (4.5,12.75);
\draw [short] (1.25,13.25) .. controls (1.25,12.5) and (1.25,12.75) .. (1.25,12.25);
\draw [short] (1.25,13.25) -- (2,12.75);
\draw [short] (1.25,12.25) -- (2,12.75);
\draw [short] (1.25,10.75) -- (1,10.5);
\draw [short] (1.25,13.25) -- (1,13);
\draw [short] (1.25,11.25) -- (1,11);
\draw [short] (1.25,13) -- (1,12.75);
\draw [short] (1.25,12.75) -- (1,12.5);
\draw [short] (1.25,12.5) -- (1,12.25);
\draw [short] (2,10.75) -- (1.25,11.25);
\draw [short] (1.25,11.25) -- (1.25,10.25);
\draw [short] (1.25,10.25) -- (2,10.75);
\draw [short] (1,10.75) -- (1.25,11);

\draw [short] (1.25,10.5) -- (1,10.25);
\end{circuitikz}
}
\end{figure}
     

		 \begin{enumerate}
	\item zero
	\item $\frac{P}{\sqrt{2}}$
	\item $P$
	\item $\sqrt{2}P$

\end{enumerate}

\item The major and minor principal stresses at a point 3 MPa and -3 MPa respectively. The maximum shear stress at the point is \hfill{(2010-CE)}

\begin{enumerate}
	\item zero
		\item 3 MPa
		\item 6 MPa
		\item 9 MPa
\end{enumerate}
	
\item The number of independent elastic constants for a linear elastic isotropic and homogeneous material is \hfill{(2010-CE)}

\begin{enumerate}
	\item 4
	\item 3
	\item 2
	\item 1
\end{enumerate}
			 
\item The effective length of a column of length L fixed against rotation and translation at one end and free at the other end is \hfill{(2010-CE)}

\begin{enumerate}
	\item 0.5 L
\item 0.7 L
\item 1.414 L
\item 2 L

\end{enumerate}

\item As per Indian standard code for practice for prestressed concrete (IS:1343-1980) the minimum grades of concerte to be used for post-tensioned and pre-tensioned structural elements are respectively \hfill{(2010-CE)}

\begin{enumerate}
\item M20 for both
 \item M40 and M30
 \item M15 and M20
 \item M30 and M40
\end{enumerate}


\item A solid circular shaft of diameter $d$ and length $L$ is fixed at one end and free at the other end. A torque $T$ is applied at the free end. The shear modulus of the material is $G$. The angle of twist at the free end is \hfill{(2010-CE)}

\begin{enumerate}
				\item $\frac{16TL}{\pi d^{4} G}$

\item $\frac{32TL}{\pi d^{4} G}$

\item $\frac{64TL}{\pi d^{4} G}$

\item $\frac{128TL}{\pi d^{4} G}$

\end{enumerate}
	
\item  In a compaction test, $G, w, S$ and $e$ represent the specific gravity, water content, degree of saturation and void ratio of the soil sample, respectively. If $\gamma_{w}$ represents the unit weight of water and $\gamma_{d}$ represents the dry unit weight of the soil, the equation for zero air voids line is \hfill{(2010-CE)}

\begin{enumerate}
    \item $\gamma_{d} = \frac{G \gamma_{w}}{1+Se}     $

\item $\gamma_{d} = \frac{G \gamma_{w}}{1+Gw}     $

\item $\gamma_{d} = \frac{Gw} {1+S\gamma_{w}}     $

\item $\gamma_{d} = \frac{Gw}{1+Se}     $

\end{enumerate}
	

\item A fine grained soil has liquid limit of 60 and plastic limit of 20. As per the plasticity chart, according to IS classification, the soil is represented by the letter symbols\hfill{(2010-CE)}

\begin{enumerate}
	\item CL
	\item CI
	\item CH
	\item CL-ML

\end{enumerate}
\item Quick sand condition occurs when	\hfill{(2010-CE)}
	\begin{enumerate}
		\item the void ratio of the soil becomes 1.0
		\item the upward seepage pressure in soil becomes zero
		\item the upward seepage pressure in soil becomes equal to the saturated unit weight of the soil	
		\item the upward seepage pressure in soil becomes equal to the submerged unit weightof the soil
	\end{enumerate}

\end{enumerate}

\end{document}



