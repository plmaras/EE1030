\let\negmedspace\undefined
\let\negthickspace\undefined
\documentclass[journal,12pt,twocolumn]{IEEEtran}
\usepackage{cite}
\usepackage{amsmath,amssymb,amsfonts,amsthm}
\usepackage{algorithmic}
\usepackage{graphicx}
\usepackage{textcomp}
\usepackage{xcolor}
\usepackage{txfonts}
\usepackage{tikz}
\usepackage{circuitikz}
\usepackage{enumitem}
\usepackage{mathtools}
\usepackage{multicol}
\usepackage{gensymb}
\usepackage{comment}
\usepackage[breaklinks=true]{hyperref}
\usepackage{tkz-euclide} 
\usepackage{listings}
\usepackage{gvv}                                        
%\def\inputGnumericTable{}                                 
\usepackage[latin1]{inputenc}                                
\usepackage{color}                                            
\usepackage{array}                                            
\usepackage{longtable}                                       
\usepackage{calc}                                             
\usepackage{multirow}                                         
\usepackage{hhline}                                           
\usepackage{ifthen}                                           
\usepackage{lscape}
\usepackage{tabularx}
\usepackage{array}
\usepackage{float}



\newtheorem{theorem}{Theorem}[section]
\newtheorem{problem}{Problem}
\newtheorem{proposition}{Proposition}[section]
\newtheorem{lemma}{Lemma}[section]
\newtheorem{corollary}[theorem]{Corollary}
\newtheorem{example}{Example}[section]
\newtheorem{definition}[problem]{Definition}
\newcommand{\BEQA}{\begin{eqnarray}}
\newcommand{\EEQA}{\end{eqnarray}}
\newcommand{\define}{\stackrel{\triangle}{=}}
\theoremstyle{remark}
\newtheorem{rem}{Remark}
\begin{document}
\onecolumn
\bibliographystyle{IEEEtran}

\vspace{3cm}
\title{ 2012-ME-1-13 }
\author{Golla Shriram - AI24BTech11010}

\maketitle
%\newpage
%\bigskip

\renewcommand{\thefigure}{\theenumi}
\renewcommand{\thetable}{\theenumi}

\section{ Q.1-Q.25 carry one mark each. }
                                                                           
 \begin{enumerate}
		 
	 \item In abrasive jet machining, as the distance between the nozzle tip and the work surface increases, the material removal rate \hfill{(2012-ME)}
	 
\begin{enumerate}
    \begin{multicols}{1} 
        \item increases continuously   
	\item decreases continuously
        \item decreases, becomes stable and then increases
	\item increases, becomes stable and then decreases
	\end{multicols}
\end{enumerate}


\item Match the following metal forming processes with their associated stresses in the workpiece \hfill{(2012-ME)}
\[
\begin{array}{|c|c|c|}
\hline
\text{\textbf{Metal Forming Process}} & \text{\textbf{Type of Stress}} \\
\hline
 \text{ 1.   Coining} & S. \  \text{Compressive} \\
  \text{2.   Wire Drawing} & P.  \ \text{Tensile} \\
 \text{3.   Blanking} & Q. \   \text{Shear} \\
 \text{4.   Deep Drawing} & R. \  \text{Tensile and Compressive} \\
\hline
\end{array}
\]


	\begin{enumerate}
		    \begin{multicols}{2} 
	\item 1-S, 2-P, 3-Q, 4-R
	\item 1-S, 2-P, 3-R, 4-Q
	\item 1-P, 2-Q, 3-S, 4-R
	\item 1-P, 2-R, 3-Q, 4-S
\end{multicols}  
\end{enumerate}

\item In an interchangeable assembly, shafts of size ${25.000}^{\begin{matrix} +0.040 \\ -0.010 \end{matrix}} mm$ mate with holes of size ${25.000}^{\begin{matrix}    +0.030 \\ +0.020 \end{matrix}} mm$. The maximum interference (in $micron$s) in the assembly is \hfill{(2012-ME)} 
                 \begin{enumerate}   
                     \begin{multicols}{4} 
\item 40
\item 30
\item 20
\item 10

		     \end{multicols}                            
                         \end{enumerate}





\item During $normalizing$ process of steel, the specimen is heated \hfill{(2012-ME)}
\begin{enumerate}
\item  between the upper and lower critical temperature and cooled in still air.
\item  above the upper critical temperature and cooled in furnace.
\item  above the upper critical temperature and cooled in still air
\item  between the upper and lower critical temperature and cooled in furnace.
\end{enumerate}

\item Oil flows through a 200 $mm$ diameter horizontal cast iron pipe (friction factor, $f$ = 0.0225) of length 500 m. The volumetric flow rate is 0.2 $m^3/s$. The head loss (in $m$) due to friction is (assume $g$ = 9.81 $m/s^2$)\hfill{(2012-ME)}
\begin{enumerate}
\begin{multicols}{4} 
\item 116.18
\item 0.116
\item 18.22
\item 232.36
\end{multicols}  
\end{enumerate}


\item For an opaque surface, the absorptivity($\alpha$), transmissivity($\tau$) and reflectivity($\rho$)are related by the equation \hfill{(2012-ME)} 
                 \begin{enumerate}   
                     \begin{multicols}{4}       
\item $\rho + \alpha = \tau $
\item $\rho + \alpha + \tau = 0$
\item $\rho + \alpha = 1$
\item $\rho + \alpha = 0$

		     \end{multicols}                            
                         \end{enumerate}


\item Steam enters an adiabatic turbine operating at steady state with an enthalpy of 3251.0 $kJ/kg$ and leaves as a saturated mixture at 15 $k$Pa with quality (dryness fraction) 0.9. The enthalpies of the saturated liquid and vapor at 15 $k$Pa are $h_f$ = 225.94 $kJ/kg$ and $h_g$ = 2598.3 $kJ/kg$ respectively. The mass flow rate of steam is 10 $kg/s$. Kinetic and potential energy changes are negligible. The power output of the turbine in $MW$ is \hfill{(2012-ME)} 
  \begin{enumerate}   
 \begin{multicols}{4}          
\item 6.5
\item 8.9
\item 9.1
\item 27.0
\end{multicols}                            
   \end{enumerate}


\item The following are the data for two crossed helical gears used for speed reduction:\\
Gear I : Pitch circle diameter in the plane of rotation 80 $mm$ and helix angle 30\degree \\
Gear II : Pitch circle diameter in the plane of rotation 120 $mm$ and helix angle 22.5\degree \\
If the input speed is 1440 rpm, the output speed in rpm is \hfill{(2012-ME)} 
                 \begin{enumerate}   
 \begin{multicols}{4}                      
\item 1200
\item 900
\item 875
\item 720
\end{multicols}      \end{enumerate}


\item A solid disc of radius $r$ rolls without slipping on a horizontal floor with angular velocity $\omega$ and
angular acceleration $\alpha$. The magnitude of the acceleration of the point of contact on the disc is \hfill{(2012-ME)} 
                 \begin{enumerate}   
                     \begin{multicols}{4}    
\item $zero$
\item $r\alpha$
\item $\sqrt{(r\alpha)^{2}+(r(\omega)^{2})^{2}}$
\item $r{\omega}^2$
		     \end{multicols}                            
                         \end{enumerate}



\item A thin walled spherical shell is subjected to an internal pressure. If the radius of the shell is increased by $1\%$ and the thickness is reduced by $1\%$, with the internal pressure remaining the same, the percentage change in the circumferential (hoop) stress is \hfill{(2012-ME)} 
  \begin{enumerate} 
\begin{multicols}{4}            
   \item 0
  \item 1
 \item 1.08
   \item 2.02
\end{multicols}       
  \end{enumerate}


\item The area enclosed between the straight line $y = x$ and the parabola $y = x^{2}$ in the x-y plane is \hfill{(2012-ME)} 
                 \begin{enumerate}   
                     \begin{multicols}{4}           

\item $\frac{1}{6}$
\item $\frac{1}{4}$
\item $\frac{1}{3}$
\item $\frac{1}{2}$
		     \end{multicols}                            
                         \end{enumerate}



\item Consider the function $f(x)=|x|$ in the interval $-1\leq x \geq 1$. At the point $x = 0, f(x)$ is \hfill{(2012-ME)} 
                 \begin{enumerate}   
                     \begin{multicols}{2}                                  
		     		     \item  continuous and differentiable. 
		     \item  non-continuous and differentiable. 
		     \item  continuous and non-differentiable. 
		     \item   neither continuous nor differentiable.
		     \end{multicols}
		     \end{enumerate}

\item Which one of the following is \textbf{NOT} a decision taken during the aggregate production planning stage? \hfill{(2012-ME)} 
                 \begin{enumerate}   
                     \begin{multicols}{2}
		     \item  Scheduling of machines 
		     \item  Amount of labour to be committed
		     \item  Rate at which production should happen
		     \item  Inventory to be carried forward
		     \end{multicols}                            
                         \end{enumerate}








\end{enumerate}

\end{document}

