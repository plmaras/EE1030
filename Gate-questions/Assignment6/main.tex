\let\negmedspace\undefined
\let\negthickspace\undefined
\documentclass[journal,12pt,twocolumn]{IEEEtran}
\usepackage{cite}
\usepackage{amsmath,amssymb,amsfonts,amsthm}
\usepackage{algorithmic}
\usepackage{graphicx}
\usepackage{textcomp}
\usepackage{xcolor}
\usepackage{txfonts}
\usepackage{tikz}
\usepackage{circuitikz}
\usepackage{enumitem}
\usepackage{mathtools}
\usepackage{multicol}
%\usepackage{multicol}
\usepackage{gensymb}
\usepackage{comment}
\usepackage[breaklinks=true]{hyperref}
\usepackage{tkz-euclide} 
\usepackage{listings}
\usepackage{gvv}                                        
%\def\inputGnumericTable{}                                 
\usepackage[latin1]{inputenc}                                
\usepackage{color}                                            
\usepackage{array}                                            
\usepackage{longtable}                                       
\usepackage{calc}                                             
\usepackage{multirow}                                         
\usepackage{hhline}                                           
\usepackage{ifthen}                                           
\usepackage{lscape}
\usepackage{tabularx}
\usepackage{array}
\usepackage{float}
\usepackage{standalone}



\newtheorem{theorem}{Theorem}[section]
\newtheorem{problem}{Problem}
\newtheorem{proposition}{Proposition}[section]
\newtheorem{lemma}{Lemma}[section]
\newtheorem{corollary}[theorem]{Corollary}
\newtheorem{example}{Example}[section]
\newtheorem{definition}[problem]{Definition}
\newcommand{\BEQA}{\begin{eqnarray}}
\newcommand{\EEQA}{\end{eqnarray}}
\newcommand{\define}{\stackrel{\triangle}{=}}
\theoremstyle{remark}
\newtheorem{rem}{Remark}
\begin{document}
\onecolumn
\bibliographystyle{IEEEtran}

\vspace{3cm}
\title{ 2014-EE-1-13 }
\author{Golla Shriram - AI24BTech11010}

\maketitle
%\newpage
%\bigskip

\renewcommand{\thefigure}{\theenumi}
\renewcommand{\thetable}{\theenumi}

                                                                           
 \begin{enumerate}[start=53]
		 
	 \item The overcurrent relays for the line protection and loads connected at the buses are shown in the figure \hfill{(2014-EE)}
     \begin{center}
\documentclass{standalone}
\usepackage{amsmath, tikz}
\usepackage{circuitikz}

\begin{document}


\centering
\resizebox{0.5\textwidth}{!}{%
\begin{circuitikz}
\tikzstyle{every node}=[font=\large]
\draw [ line width=2pt](1.75,10.75) to[short] (1.75,8.5);
\draw [ line width=1.6pt](0.25,9.5) to[short] (8,9.5);

\draw[ ultra thick ] (2.21,9.3) -- (2.3,9.25) -- (2.21,9.2) -- cycle;
\draw [ line width=2pt](5,10.75) to[short] (5,8.5);
\draw [ line width=2pt](7.5,10.75) to[short] (7.5,8.5);
\draw [ line width=0.8pt](5,8.75) to[short] (5.5,8.75);
\draw [ line width=1.2pt](1.75,8.75) to[short] (2.25,8.75);
\draw [line width=1.2pt, ->, >=Stealth] (8,9.5) -- (8,7.75);

\draw[ ultra thick ] (5.51,9.3) -- (5.6,9.25) -- (5.51,9.2) -- cycle;
\node [font=\large] at (2.75,9) {$R_A$};
\node [font=\large] at (6,9) {$R_B$};
\draw [line width=1.2pt, ->, >=Stealth] (2.25,8.75) -- (2.25,7.75);
\draw [line width=1.2pt, ->, >=Stealth] (5.5,8.75) -- (5.5,7.75);
\node [font=\Large] at (1.75,11.25) {A};
\node [font=\Large] at (5,11.25) {B};
\node [font=\Large] at (7.5,11.25) {C};
\node [font=\large] at (2.75,7.25) {\text{300A}};
\node [font=\large] at (5.75,7.25) {\text{200A}};
\node [font=\large] at (8.25,7.25) {\text{100A}};
\draw [ line width=1pt](-0.5,9.5) to[sinusoidal voltage source, sources/symbol/rotate=auto] (0.25,9.5);
\end{circuitikz}
}%




\end{document}

\end{center}


	 
The relays are IDMT in nature having the characteristic
$$  t_{op} = \frac{0.14 * Time Multiplier Setting}{\brak{Plug Setting Multiplier}^{0.02}-1}        $$
The maximum and minimum fault currents at bus B are 2000 A and 500 A respectively. Assuming
the time multiplier setting and plug setting for relay RB to be 0.1 and 5A respectively, the operating
time of RB  (in seconds) is \underline{\hspace{2.5 cm}}.

\item For the given system, it is desired that the system be stable. The minimum value of $\alpha$ for this
condition is \underline{\hspace{2.5 cm}}. \hfill{(2014-EE)}

\begin{center}
\documentclass{standalone}
\usepackage{amsmath, tikz}
\usepackage{circuitikz}

\begin{document}
\centering
\resizebox{0.9\textwidth}{!}{%
\begin{circuitikz}
\tikzstyle{every node}=[font=\normalsize]


\draw  (6.25,14) circle (0.75cm);
\draw [->, >=Stealth] (7,14) -- (8.25,14);
\draw  (8.25,15) rectangle (15.75,13.25);
\draw [->, >=Stealth] (4.5,14) -- (5.5,14);
\draw [->, >=Stealth] (6.25,12) .. controls (6.25,12.5) and (6.25,13) .. (6.25,13.25) ;
\draw [->, >=Stealth] (15.75,14) -- (17.75,14);
\draw [short] (6.25,12) -- (16.75,12);
\draw [short] (16.75,12) -- (16.75,14);
\node [font=\normalsize] at (4,14) {$R(s)$};
\node [font=\normalsize] at (18.5,14) {$C(s)$};
\node [font=\LARGE] at (12,14) {$\frac{(s+\alpha)}{s^{3}+ (1+\alpha )s^2 + (\alpha-1)s + (1-\alpha)}$};
\node [font=\large] at (5.75,14) {+};
\node [font=\large] at (6.25,13.5) {-};
\draw [short] (5.75,14.5) -- (6.75,13.5);
\draw [short] (5.75,13.5) -- (6.75,14.5);
\end{circuitikz}
}%

\end{document}

\end{center}




\item The Bode magnitude plot of the transfer function $G(s) =\frac{K\brak{1+0.5s}\brak{1+\alpha s} }{s\brak{1+\frac{s}{8}}\brak{1+bs}\brak{1+\frac{s}{36}} }$ is shown below:
Note that -6 dB/octave $=$ - 20 dB/decade. The value of $\frac{a}{bK}$ \underline{\hspace{2.5 cm}}.  \hfill{(2014-EE)} 


\begin{center}
\documentclass{standalone}
\usepackage{amsmath, tikz}
\usepackage{circuitikz}

\begin{document}
\centering
\resizebox{0.8\textwidth}{!}{%
\begin{circuitikz}
\tikzstyle{every node}=[font=\normalsize]
\draw [->, >=Stealth] (3.75,4.25) -- (3.75,12);
\draw [->, >=Stealth] (3.75,4.25) -- (23.75,4.25);
\draw [->, >=Stealth] (20,3.5) -- (20.75,3.5);
\draw [->, >=Stealth] (3,9.75) -- (3,10.5);
\node [font=\LARGE] at (24,3.5) {$\omega$ rads/s};
\node [font=\large] at (2.5,8.5) {db};
\draw [line width=1pt, short] (3.75,10.25) -- (5.75,8.5);
\draw [line width=1pt, short] (5.75,8.5) -- (8.25,8.5);
\draw [line width=1pt, short] (8.25,8.5) -- (10.75,10.25);
\draw [line width=1pt, short] (10.75,10.25) -- (16.5,10.25);
\draw [line width=1pt, short] (16.5,10.25) -- (18.5,8.75);
\draw [line width=1pt, short] (18.5,8.75) -- (20.5,6.25);
\draw [line width=0.5pt, dashed] (5.75,8.5) -- (5.75,4.25);
\draw [line width=0.5pt, dashed] (8.25,8.5) -- (8.25,4.25);
\draw [line width=0.5pt, dashed] (5.75,8.5) -- (10.75,4.25);
\draw [line width=0.5pt, dashed] (10.75,10.25) -- (10.75,4.25);
\draw [line width=0.5pt, dashed] (16.5,10.25) -- (16.5,4.25);
\draw [line width=0.5pt, dashed] (18.5,8.75) -- (18.5,4.5);
\draw [line width=0.5pt, dashed] (18.5,8.75) -- (21.25,7);
\node [font=\Large] at (4.5,3.75) {0.01};
\node [font=\Large] at (5.75,3.75) {2};
\node [font=\Large] at (8.25,3.75) {4};
\node [font=\Large] at (10.75,3.75) {8};
\node [font=\large] at (7,8.75) {0db/Octave};
\node [font=\Large] at (18.5,3.75) {36};
\node [font=\large] at (14,10.75) {0db/Octave};
\node [font=\large] at (5.75,10.25) {-6db/Octave};
\node [font=\large] at (9.25,10.25) {6db/Octave};
\node [font=\Large] at (16.25,3.75) {24};
\node [font=\large] at (18.5,10.25) {-6db/Octave};
\node [font=\large] at (20,5.75) {-12db/Octave};
\node [font=\Large] at (3.25,4.5) {0};
\end{circuitikz}
}%
\end{document}

\end{center}


\item A system matrix is given as follows.


$$ A =    \begin{bmatrix}
    0     & 1  & -1 \\
    -6    &  -11 & 6 \\
  -6     &  -11 & 5
\end{bmatrix}$$

The absolute value  of the ratio of the maximum eigenvalue to the minimum eigenvalue is \underline{\hspace{2.5 cm}}.  \hfill{(2014-EE)}

\item The reading of the voltmeter (rms) in volts, for the circuit shown in the figure is \underline{\hspace{2.5 cm}}  \hfill{(2014-EE)}


\begin{center}
\documentclass{standalone}
\usepackage{amsmath, tikz}
\usepackage{circuitikz}

\begin{document}

\centering
\resizebox{0.4\textwidth}{!}{%
\begin{circuitikz}
\tikzstyle{every node}=[font=\LARGE]
\draw (5.25,4.25) to[short] (5.5,4.25);
\draw (1.25,13.5) to[european resistor] (7.25,13.5);
\draw (5.5,12) to[L ] (5.5,9);
\draw (9,9) to[L ] (9,5);
\draw (9,12) to[C] (9,9);
\draw (5.5,8.5) to[C] (5.5,5);
\draw (5.5,9) to[short] (5.5,7.25);
\draw (5.5,12) to[short] (7.5,12);
\draw (7.25,12) to[short] (9,12);
\draw (5.5,8.5) to[short] (6.75,8.5);
\draw (5.5,5) to[short] (7.5,5);
\draw (5.5,5) to[short] (9,5);
\node at (1.25,4.25) [circ] {};
\node at (1.25,4.25) [circ] {};
\draw (1.25,4.25) to[short] (7.25,4.25);
\draw (7.25,4.25) to[short] (7.25,5);
\node at (1.25,13.5) [circ] {};
\draw (7.25,12) to[short] (7.25,13.5);
\draw  (7.25,8.5) ellipse (0.5cm and 0.75cm);
\draw (7.75,8.5) to[short] (9,8.5);
\node [font=\LARGE] at (7.25,8.5) {V};
\node [font=\LARGE] at (4,14.25) {R = 0.5 $\Omega$};
\draw [->, >=Stealth] (1.25,9.75) .. controls (1.25,11.5) and (1.25,11.5) .. (1.25,13.25) ;
\draw [->, >=Stealth] (1.25,7.75) -- (1.25,4.5);
\node [font=\LARGE] at (10.75,10.5) {1/j$\Omega$};
\node [font=\LARGE] at (1.25,8.75) {100 $sin(\omega t)$};
\node [font=\LARGE] at (4.25,6.75) {1/j$\Omega$};
\node [font=\LARGE] at (4.5,10.5) {1j$\Omega$};
\node [font=\LARGE] at (11,7) {1j$\Omega$};
\end{circuitikz}
}%

\end{document}

\end{center}




\item The dc current flowing in a circuit is measured by two ammeters, one PMMC and another
electrodynamometer type, connected in series. The PMMC meter contains 100 turns in the coil, the
flux density in the air gap is 0.2 $Wb/m^2$, and the area of the coil is 80 $mm^2$. The electrodynamometer
ammeter has a change in mutual inductance with respect to deflection of 0.5 mH/deg. The spring
constants of both the meters are equal. The value of current, at which the deflections of the two
meters are same, is \underline{\hspace{2.5 cm}}.  \hfill{(2014-EE)}
\\

\item Given that the op-amps in the figure are ideal, the output voltage $V_0$ is 
\begin{center}
\documentclass{standalone}
\usepackage{amsmath, tikz}
\usepackage{circuitikz}

\begin{document}

\centering
\resizebox{0.5\textwidth}{!}{%
\begin{circuitikz}
\tikzstyle{every node}=[font=\Huge]

\draw (8.25,10) to[european resistor] (10.5,10);
\draw (8.25,7.5) to[european resistor] (8.25,5.75);
\draw (1.75,7.45) to[european resistor] (4,7.49);
\draw (1.75,8.55) to[european resistor] (4,8.5);
\draw (1.75,8.75) to[european resistor] (1.75,7.25);
\draw (3.25,5.75) node[op amp](opamp1){};
\draw (opamp1.-) to[short] (1.75,6.25);
\draw (4.45,5.75) to[short](4.75,5.75);
\node at (1.75,10.5) [circ] {};



\draw (3.25,10) node[op amp,scale=1, yscale=-1 ] (opamp2) {};
\draw (opamp2.+) to[short] (1.75,10.5);
\draw (opamp2.-) to[short] (1.75,9.5);
\draw (4.45,10) to[short](4.75,10);


\draw (9.5,8) node[op amp ] (opamp3) {\text{}};
\draw (opamp3.-) to[short] (8,8.5);
\draw (opamp3.+) to[short] (8,7.5);
\draw (10.7,8) to[short](11,8);


\draw (6.25,8.5) to[european resistor] (8.25,8.5);
\draw (5.75,7.5) to[european resistor] (8.25,7.5);
\draw (1.75,9.5) to[short] (1.75,8.75);
\draw (1.75,6.25) to[short] (1.75,7.25);
\draw (3.5,8.5) to[short] (6.25,8.5);
\draw (3.75,7.5) to[short] (6,7.5);
\draw (8.25,5.75) to (8.25,5.5) node[ground]{};
\draw (4.75,10) to[short] (4.75,8.5);
\draw (4.75,7.5) to[short] (4.75,5.75);
\draw (8.25,8.5) to[short] (8.25,10);
\draw (10.5,8) to[short] (10.5,10);
\node at (11,8) [circ] {};
\node at (2,5.25) [circ] {};
\node [font=\normalsize] at (1.5,10.75) {$V_{2}$};
\node [font=\normalsize] at (1.25,5.5) {$V_{1}$};
\node [font=\normalsize] at (1.25,8) {2R};
\node [font=\normalsize] at (3.25,9) {R};
\node [font=\small] at (11.25,8) {$V_{0}$};
\node [font=\normalsize] at (9.25,10.5) {R};
\node [font=\normalsize] at (7.5,7) {R};
\node [font=\normalsize] at (7.25,9) {R};
\node [font=\normalsize] at (3,7) {R};
\node [font=\normalsize] at (8.75,6.5) {R};
\end{circuitikz}
}%



\end{document}

\end{center}
	\begin{enumerate}
		    \begin{multicols}{2} 
	\item $\brak{V_1-V_2}$
	\item   $\brak{V_1-V_2} /2$
	\item  $2\brak{V_1-V_2}$
	\item   $\brak{V_1+V_2}$
\end{multicols}  
\end{enumerate}
\item Which of the following logic circuits is a realization of the function F whose Karnaugh map is
shown in figure \hfill{(2014-EE)}

\begin{center}
\documentclass{standalone}
\usepackage{amsmath, tikz}
\usepackage{circuitikz}

\begin{document}
\centering
\resizebox{0.4\textwidth}{!}{%
\begin{circuitikz}
\tikzstyle{every node}=[font=\LARGE]
\draw (0.25,13) to[short] (7.75,13);
\draw (0.25,11) to[short] (7.75,11);
\draw (0.25,9) to[short] (7.75,9);
\draw (0.25,13) to[short] (0.25,9);
\draw (2.25,13) to[short] (2.25,9);
\draw (4.25,13) to[short] (4.25,9);
\draw (6,13) to[short] (6,9);
\draw (7.75,13) to[short] (7.75,9);
\draw [short] (0.25,13) -- (-2.5,15.25);
\node [font=\Large] at (-0.25,12) {0};
\node [font=\LARGE] at (4.25,14.75) {AB};
\node [font=\Large] at (-1.25,10.75) {C};
\node [font=\LARGE] at (1.25,11.75) {1};
\node [font=\LARGE] at (-0.25,10) {1};
\node [font=\LARGE] at (3.25,10) {1};
\node [font=\LARGE] at (3.25,11.75) {1};
\node [font=\LARGE] at (5,10) {1};
\node [font=\LARGE] at (6.75,13.5) {11};
\node [font=\LARGE] at (1.25,13.5) {00};
\node [font=\LARGE] at (5,13.5) {10};
\node [font=\LARGE] at (3.25,13.5) {01};
\end{circuitikz}
}%




\end{document}

\end{center}

\begin{enumerate}
       \begin{multicols}{2} 
\item    \documentclass{standalone}
\usepackage{amsmath, tikz}
\usepackage{circuitikz}

\begin{document}

\centering

\resizebox{0.4\textwidth}{!}{%
\begin{circuitikz}
\tikzstyle{every node}=[font=\LARGE]

 \draw (4.5,10.25) to[short] (4.75,10.25);
\draw (4.5,9.75) to[short] (4.75,9.75);
\draw (4.75,10.25) node[ieeestd nand port, anchor=in 1, scale=0.89](port){} (port.out) to[short] (6.5,10);
 \draw (5,8.25) to[short] (5.25,8.25);
 \draw (5,7.75) to[short] (5.25,7.75);
\draw (5.25,8.25) node[ieeestd and port, anchor=in 1, scale=0.89](port){} (port.out) to[short] (7,8);
\draw (7.75,10) to[short] (7.75,9.25);
 \draw (7.75,8.75) to[short] (8.5,8.75);
\draw (7.75,9.25) to[short] (8.25,9.25);
\draw (7.75,8.75) to[short] (7.75,8);
\draw (7,8) to[short] (7.75,8);
\draw (6.5,10) to[short] (7.75,10);

\draw (4.5,10.25) to[short] (4.5,9.75);
\draw (5,8.25) to[short] (4.25,8.25);
\draw (5,7.75) to[short] (4.25,7.75);
 \draw (4,10) to[short] (4.5,10);
\node [font=\large] at (3.75,10) {A};
\node [font=\large] at (4,8.25) {B};
\node [font=\large] at (4,7.75) {C};


 \draw (8.25,9.25) to[short] (8.5,9.25);
\node at (10.45,9) [circ] {};

\draw (8.5,9.25) node[ieeestd and port, anchor=in 1, scale=0.89](port){} (port.out) to[short] (10.25,9);
\end{circuitikz}
}%




\end{document}



\item    \documentclass{standalone}
\usepackage{amsmath, tikz}
\usepackage{circuitikz}

\begin{document}

\centering
\resizebox{0.4\textwidth}{!}{%
\begin{circuitikz}
\tikzstyle{every node}=[font=\LARGE]





\draw (4.5,10.25) to[short] (4.75,10.25);
\draw (4.5,9.75) to[short] (4.75,9.75);
\draw (4.75,10.25) node[ieeestd nand port, anchor=in 1, scale=0.89](port){} (port.out) to[short] (6.5,10);
\draw (5,8.25) to[short] (5.25,8.25);
\draw (5,7.75) to[short] (5.25,7.75);
\draw (5.25,8.25) node[ieeestd and port, anchor=in 1, scale=0.89](port){} (port.out) to[short] (7,8);
\draw (7.75,10) to[short] (7.75,9.25);
\draw (7.75,8.75) to[short] (8.5,8.75);
\draw (7.75,9.25) to[short] (8.25,9.25);
\draw (7.75,8.75) to[short] (7.75,8);
\draw (7,8) to[short] (7.75,8);
\draw (6.5,10) to[short] (7.75,10);
\draw (4.5,10.25) to[short] (4.5,9.75);
\draw (5,8.25) to[short] (4.25,8.25);
\draw (5,7.75) to[short] (4.25,7.75);
\draw (4,10) to[short] (4.5,10);

\node [font=\large] at (3.75,10) {A};
\node [font=\large] at (4,8.25) {B};
\node [font=\large] at (4,7.75) {C};

\node at (10.4,9) [circ] {};

\draw (8.25,9.25) to[short] (8.5,9.25);
\draw (8.25,8.75) to[short] (8.5,8.75);
\draw (8.5,9.25) node[ieeestd or port, anchor=in 1, scale=0.89](port){} (port.out) to[short] (10.25,9);
\end{circuitikz}
}%


\end{document}


\item  \documentclass{standalone}
\usepackage{amsmath, tikz}
\usepackage{circuitikz}

\begin{document}

\centering
\resizebox{0.4\textwidth}{!}{%
\begin{circuitikz}
\tikzstyle{every node}=[font=\large]

\draw (7.75,15.5) to[short] (8,15.5);
\draw (7.75,15) to[short] (8,15);
\draw (8,15.5) node[ieeestd nand port, anchor=in 1, scale=0.89](port){} (port.out) to[short] (9.75,15.25);
\draw (7.75,13.75) to[short] (8,13.75);
\draw (7.75,13.25) to[short] (8,13.25);
\draw (8,13.75) node[ieeestd nand port, anchor=in 1, scale=0.89](port){} (port.out) to[short] (9.75,13.5);
\draw (10.5,15.25) to[short] (10.75,15.25);
\draw (10.5,14.75) to[short] (10.75,14.75);
\draw (10.57,15.25) node[ieeestd and port, anchor=in 1, scale=0.89](port){} (port.out) to[short] (12.3,15);
\draw (9.75,15.25) to[short] (10.5,15.25);
\draw (12.5,15) to[short] (12.5,13.5);
\draw (7.75,15.5) to[short] (7.75,15);
\draw (7.75,13.75) to[short] (7.75,13.25);
\draw (9.75,13.5) to[short] (10.5,13.5);
\draw (10.5,13.5) to[short] (10.5,14.75);
\draw (12,13) to[short] (12.75,13);
\draw (7.75,15.25) to[short] (7,15.25);
\draw (7.75,13.5) to[short] (7,13.5);
\draw (7.75,13.25) to[short] (7.75,12);
\draw (7.75,12) to[short] (9.5,12);
\draw (7.5,11.5) to[short] (9.25,11.5);
\draw (9.25,12) to[short] (9.75,12);
\draw (9.25,11.5) to[short] (9.75,11.5);
\draw (9.75,12) node[ieeestd and port, anchor=in 1, scale=0.89](port){} (port.out) to[short] (12,11.75);
\draw (12.5,13.5) to[short] (12.75,13.5);
\draw (12.5,13) to[short] (12.75,13);
\draw (12.75,13.5) node[ieeestd or port, anchor=in 1, scale=0.89](port){} (port.out) to[short] (14.5,13.25);
\draw [line width=0.7pt, short] (12,11.75) -- (12,13);
\node at (14.65,13.25) [circ] {};




\node [font=\large] at (6.75,15.25) {A};
\node [font=\large] at (7,11.5) {B};
\node [font=\large] at (6.75,13.5) {C};
\end{circuitikz}
}%



\end{document}

\item    \documentclass{standalone}
\usepackage{amsmath, tikz}
\usepackage{circuitikz}

\begin{document}

\centering
\resizebox{0.4\textwidth}{!}{%
\begin{circuitikz}
\tikzstyle{every node}=[font=\LARGE]



\draw (2.75,11.75) to[short] (3,11.75);
\draw (2.75,11.25) to[short] (3,11.25);
\draw (3,11.75) node[ieeestd nand port, anchor=in 1, scale=0.89](port){} (port.out) to[short] (4.75,11.5);
\draw (3.75,9.5) to[short] (4,9.5);
\draw (3.75,9) to[short] (4,9);
 \draw (4,9.5) node[ieeestd and port, anchor=in 1, scale=0.9](port){} (port.out) to[short] (5.75,9.25);
\draw (6.75,11.25) to[short] (6.75,10.5);
  \draw (3.75,10.5) to[short] (3.75,9.5);
 \draw (6.75,10) to[short] (6.75,9.25);
 \draw (5.75,9.25) to[short] (6.75,9.25);

 \draw (2.75,11.75) to[short] (2.75,11.25);
 \draw (2.25,11.5) to[short] (2.75,11.5);

\node [font=\large] at (2,11.5) {A};
\node [font=\large] at (2.2,9) {B};
\node [font=\large] at (2,10.5) {C};


 \draw (6.75,10.5) to[short] (7,10.5);
\draw (6.75,10) to[short] (7,10);
  \draw (7,10.5) node[ieeestd or port, anchor=in 1, scale=0.89](port){} (port.out) to[short] (8.75,10.25);
\node at (9,10.25) [circ] {};
\draw (4.75,11.5) to[short] (5,11.5);
 \draw (4.75,11) to[short] (5,11);
\draw (4.85,11.5) node[ieeestd and port, anchor=in 1, scale=0.89](port){} (port.out) to[short] (6.52 ,11.25);
\draw (4.75,11) to[short] (4.75,10.5);
\draw (2.5,10.5) to[short] (4.75,10.5);
 \draw (2.5,9) to[short] (4,9);
\end{circuitikz}
}%


\end{document}

\end{multicols}  
\end{enumerate}



\item In the figure shown, assume the op-amp to be ideal. Which of the alternatives gives the correct Bode plots for the transfer function $\frac{V_0 \brak{\omega}}{V_i\brak{\omega}}$\hfill{(2014-EE)} 


\documentclass{standalone}
\usepackage{amsmath, tikz}
\usepackage{circuitikz}

\begin{document}
\centering
\resizebox{0.35\textwidth}{!}{%
\begin{circuitikz}
\tikzstyle{every node}=[font=\large]
\draw (6.75,9) node[op amp,scale=1, yscale=-1 ] (opp2) {};
\draw (opp2.+) to[short] (5.25,9.5);
\draw  (opp2.-) to[short] (5.25,8.5);
\draw (7.95,9) to[short](8.25,9);
\draw (3,9.5) to[R] (4.75,9.5);
\draw (5.5,7) to[R] (7.5,7);
\draw (4.75,8.25) to[C] (4.75,9.5);
\draw (6.75,9.5) to[short] (6.75,10.25);
\draw (6.75,8.5) to[short] (6.75,7.75);
\node [font=\large] at (7.25,9.75) {$+V_{cc}$};
\node [font=\large] at (7.25,8) {$-V_{cc}$};
\node [font=\large] at (8.5,9) {$V_{0}$};
\draw (5.25,8.5) to[short] (5.25,7);
\draw (5.25,7) to[short] (5.75,7);
\draw (7.75,9) to[short] (7.75,7);
\draw (7.25,7) to[short] (7.75,7);
\draw (7.5,7) to[short] (7.75,7);
\node [font=\normalsize] at (3.75,10) {$1k\Omega$}; % Fixed the symbol for ohm
\node [font=\normalsize] at (4,8.5) {$1\mu F$};
\node [font=\large] at (2.5,9.5) {$V_{i}$};
\draw (4.75,9.5) to[short] (5.25,9.5);
\draw (4.75,8.5) to (4.75,8.25) node[ground] {}; % Corrected ground node
\node [font=\normalsize] at (6.5,6.75) {$R_{f}$};
\end{circuitikz}
}%

\end{document}


\begin{enumerate}
   
\item    \documentclass{standalone}
\usepackage{amsmath, tikz}
\usepackage{circuitikz}

\begin{document}

\centering
\resizebox{0.6\textwidth}{!}{%
\begin{circuitikz}
\tikzstyle{every node}=[font=\LARGE]
\draw [<->, >=Stealth,line width=1pt] (2.5,16.5) -- (2.5,5.5);
\draw [<->, >=Stealth,line width=1pt] (-0.25,10) -- (9.25,10);
\draw [<->, >=Stealth,line width=1pt] (14.5,16.5) -- (14.5,5.5);
\draw [<->, >=Stealth,line width=1pt] (11.75,10) -- (20.75,10);
\draw [short] (2.25,9.25) -- (2.75,9.25);
\draw [short] (2.25,8.25) -- (2.75,8.25);
\draw [short] (2.25,7.25) -- (2.75,7.25);
\draw [line width=2pt, short] (2.5,10) -- (5.25,10);
\draw [line width=2pt, short] (5.25,10) -- (7.25,6.5);
\draw [line width=0.6pt, short] (15,10) .. controls (19.75,10) and (17,8) .. (20.75,8.25);
\draw [line width=0.6pt, short] (14.25,9.25) -- (14.75,9.25);
\draw [line width=0.6pt, short] (14.25,8.25) -- (14.75,8.25);
\draw [line width=0.6pt, short] (14.25,7.25) -- (14.75,7.25);
\draw [line width=0.6pt, short] (15.5,10.25) -- (15.5,9.75);
\draw [line width=0.6pt, short] (4,10.25) .. controls (4,10) and (4,10) .. (4,9.75);
\draw [line width=0.6pt, short] (16.75,10.25) -- (16.75,9.75);
\draw [line width=0.6pt, short] (19.5,10.25) -- (19.5,9.75);
\draw [line width=0.6pt, short] (18,10.25) -- (18,9.75);
\draw [line width=0.6pt, short] (5.25,10.25) -- (5.25,9.5);
\draw [line width=0.6pt, short] (3.25,10.25) -- (3.25,9.75);
\node [font=\LARGE] at (13,9.25) {-$\pi$/4};
\node [font=\LARGE] at (13,8.25) {-$\pi$/2};
\node [font=\LARGE] at (13.25,14.75) {$\phi$};
\node [font=\LARGE] at (14,10.75) {0};
\node [font=\LARGE] at (14.75,10.5) {1};
\node [font=\LARGE] at (15.5,10.75) {10};
\node [font=\LARGE] at (16.75,10.75) {$10^2$};
\node [font=\LARGE] at (19.75,10.75) {$\omega$};
\node [font=\LARGE] at (18,10.75) {$10^3$};
\node [font=\LARGE] at (1.5,9.25) {-10};
\node [font=\LARGE] at (-1.75,14.25) {$20log(| \frac{V_0(\omega)}{V_i(\omega)} |)$};
\node [font=\LARGE] at (5.25,11) {$10^3$};
\node [font=\LARGE] at (3.25,9.25) {10};
\node [font=\LARGE] at (4.25,9.25) {$10^2$};
\node [font=\LARGE] at (1.5,7.25) {-30};
\node [font=\LARGE] at (1.5,8.25) {-20};
\node [font=\LARGE] at (2.75,10.5) {1};
\node [font=\LARGE] at (2,10.5) {0};
\node [font=\LARGE] at (-0.5,6.5) {$10^3$};
\node [font=\LARGE] at (8,9.25) {$\omega$};
\end{circuitikz}
}%




\end{document}


\item    \documentclass{standalone}
\usepackage{amsmath, tikz}
\usepackage{circuitikz}

\begin{document}

\centering
\resizebox{0.6\textwidth}{!}{%
\begin{circuitikz}
\tikzstyle{every node}=[font=\LARGE]
\draw [<->, >=Stealth,line width=1pt] (2.5,16.5) -- (2.5,5.5);
\draw [<->, >=Stealth,line width=1pt] (-0.25,10) -- (9.25,10);
\draw [<->, >=Stealth,line width=1pt] (14.5,16.5) -- (14.5,5.5);
\draw [<->, >=Stealth,line width=1pt] (11.75,10) -- (20.75,10);
\draw [short] (2.25,9.25) -- (2.75,9.25);
\draw [short] (2.25,8.25) -- (2.75,8.25);
\draw [short] (2.25,7.25) -- (2.75,7.25);
\draw [line width=2pt, short] (2.5,10) -- (5.25,10);
\draw [line width=2pt, short] (5.25,10) -- (7.25,6.5);
\draw [line width=0.6pt, short] (14.5,11.75) .. controls (19.75,11.75) and (16,9) .. (20.5,9);
\draw [line width=0.6pt, short] (14.25,9.25) -- (14.75,9.25);
\draw [line width=0.6pt, short] (14.25,8.25) -- (14.75,8.25);
\draw [line width=0.6pt, short] (14.25,7.25) -- (14.75,7.25);
\draw [line width=0.6pt, short] (15.5,10.25) -- (15.5,9.75);
\draw [line width=0.6pt, short] (4,10.25) .. controls (4,10) and (4,10) .. (4,9.75);
\draw [line width=0.6pt, short] (16.75,10.25) -- (16.75,9.75);
\draw [line width=0.6pt, short] (19.5,10.25) -- (19.5,9.75);
\draw [line width=0.6pt, short] (18,10.25) -- (18,9.75);
\draw [line width=0.6pt, short] (5.25,10.25) -- (5.25,9.5);
\draw [line width=0.6pt, short] (3.25,10.25) -- (3.25,9.75);
\node [font=\LARGE] at (13,9.25) {-$\pi$/4};
\node [font=\LARGE] at (13,8.25) {-$\pi$/2};
\node [font=\LARGE] at (14,14.75) {$\phi$};
\node [font=\LARGE] at (14,10.75) {0};
\node [font=\LARGE] at (14.75,10.5) {1};
\node [font=\LARGE] at (15.5,9.5) {10};
\node [font=\LARGE] at (16.75,8.5) {$10^2$};
\node [font=\LARGE] at (20,9.5) {$\omega$};
\node [font=\LARGE] at (18.75,10.75) {$10^3$};
\node [font=\LARGE] at (1.5,9.25) {-10};
\node [font=\LARGE] at (-1.75,14.25) {$20log(| \frac{V_0(\omega)}{V_i(\omega)} |)$};
\node [font=\LARGE] at (5.25,11) {$10^3$};

\node [font=\LARGE] at (3.25,9.25) {10};
\node [font=\LARGE] at (4.25,9.25) {$10^2$};
\node [font=\LARGE] at (1.5,7.25) {-30};
\node [font=\LARGE] at (1.5,8.25) {-20};
\node [font=\LARGE] at (2.75,10.5) {1};
\node [font=\LARGE] at (2,10.5) {0};
\node [font=\LARGE] at (-0.5,6.5) {$10^3$};
\node [font=\LARGE] at (8,9.25) {$\omega$};
\node [font=\LARGE] at (13.25,13.5) {$\pi$/2};
\node [font=\LARGE] at (13.25,11.75) {$\pi$/4};
\draw [line width=0.6pt, short] (14.25,11.75) -- (14.75,11.75);
\draw [line width=0.6pt, short] (14.25,13.5) -- (15,13.5);
\end{circuitikz}
}%


\end{document}



\item  \documentclass{standalone}
\usepackage{amsmath, tikz}
\usepackage{circuitikz}

\begin{document}
\centering
\resizebox{0.6\textwidth}{!}{%
\begin{circuitikz}
\tikzstyle{every node}=[font=\Large]
\draw [<->, >=Stealth,line width=1pt] (2.5,16.5) -- (2.5,5.5);
\draw [<->, >=Stealth,line width=1pt] (-0.25,10) -- (9.25,10);
\draw [<->, >=Stealth,line width=1pt] (14.5,16.5) -- (14.5,5.5);
\draw [<->, >=Stealth,line width=1pt] (11.75,10) -- (20.75,10);
\draw [short] (2.25,9.25) -- (2.75,9.25);
\draw [short] (2.25,8.25) -- (2.75,8.25);
\draw [short] (2.25,7.25) -- (2.75,7.25);
\draw [line width=2pt, short] (2.5,10) -- (4,10);
\draw [line width=2pt, short] (4,10) -- (6,6.5);
\draw [line width=0.6pt, short] (14.5,10) .. controls (19.75,10.25) and (16,7.5) .. (20.5,7.75);
\draw [line width=0.6pt, short] (14.25,9.25) -- (14.75,9.25);
\draw [line width=0.6pt, short] (14.25,8.25) -- (14.75,8.25);
\draw [line width=0.6pt, short] (14.25,7.25) -- (14.75,7.25);
\draw [line width=0.6pt, short] (15.5,10.25) -- (15.5,9.75);
\draw [line width=0.6pt, short] (4,10.25) .. controls (4,10) and (4,10) .. (4,9.75);
\draw [line width=0.6pt, short] (16.75,10.25) -- (16.75,9.75);
\draw [line width=0.6pt, short] (19.5,10.25) -- (19.5,9.75);
\draw [line width=0.6pt, short] (18,10.25) -- (18,9.75);
\draw [line width=0.6pt, short] (5.25,10.25) -- (5.25,9.5);
\draw [line width=0.6pt, short] (3.25,10.25) -- (3.25,9.75);
\node [font=\LARGE] at (13,9.25) {-$\pi$/4};
\node [font=\LARGE] at (13,8.25) {-$\pi$/2};
\node [font=\LARGE] at (14,14.75) {$\phi$};
\node [font=\LARGE] at (14,10.75) {0};
\node [font=\Large] at (14.75,9.75) {1};
\node [font=\LARGE] at (15.5,9.5) {10};
\node [font=\LARGE] at (17,11) {$10^2$};
\node [font=\LARGE] at (20,9.5) {$\omega$};
\node [font=\LARGE] at (18.75,10.75) {$10^3$};
\node [font=\LARGE] at (1.5,9.25) {-10};
\node [font=\LARGE] at (-1.75,14.25) {$20log(| \frac{V_0(\omega)}{V_i(\omega)} |)$};
\node [font=\LARGE] at (5.5,10.75) {$10^3$};

\node [font=\LARGE] at (3.25,9.25) {10};
\node [font=\LARGE] at (4.25,10.75) {$10^2$};
\node [font=\LARGE] at (1.5,7.25) {-30};
\node [font=\LARGE] at (1.5,8.25) {-20};
\node [font=\LARGE] at (2.75,10.5) {1};
\node [font=\LARGE] at (2,10.5) {0};
\node [font=\LARGE] at (-0.5,6.5) {$10^3$};
\node [font=\LARGE] at (8,9.25) {$\omega$};
\node [font=\LARGE] at (13.25,13.5) {$\pi$/2};
\node [font=\LARGE] at (13.25,11.75) {$\pi$/4};
\draw [line width=0.6pt, short] (14.25,11.75) -- (14.75,11.75);
\draw [line width=0.6pt, short] (14.25,13.5) -- (15,13.5);
\end{circuitikz} }%


\end{document}



\item    \documentclass{standalone}
\usepackage{amsmath, tikz}
\usepackage{circuitikz}

\begin{document}

\centering
\resizebox{0.6\textwidth}{!}{%
\begin{circuitikz}
\tikzstyle{every node}=[font=\LARGE]
\draw [<->, >=Stealth,line width=1pt] (2.5,16.5) -- (2.5,5.5);
\draw [<->, >=Stealth,line width=1pt] (-0.25,10) -- (9.25,10);
\draw [<->, >=Stealth,line width=1pt] (14.5,16.25) -- (14.5,5.25);
\draw [<->, >=Stealth,line width=1pt] (11.75,10) -- (20.75,10);
\draw [short] (2.25,9.25) -- (2.75,9.25);
\draw [short] (2.25,8.25) -- (2.75,8.25);
\draw [short] (2.25,7.25) -- (2.75,7.25);
\draw [line width=2pt, short] (2.5,10) -- (4,10);
\draw [line width=2pt, short] (4,10) -- (6.75,14);
\draw [line width=1pt, short] (14.25,13) .. controls (21,12.5) and (15.75,10.5) .. (20,10);
\draw [line width=0.6pt, short] (14.25,9.25) -- (14.75,9.25);
\draw [line width=0.6pt, short] (14.25,8.25) -- (14.75,8.25);
\draw [line width=0.6pt, short] (14.25,7.25) -- (14.75,7.25);
\draw [line width=0.6pt, short] (15.5,10.25) -- (15.5,9.75);
\draw [line width=0.6pt, short] (4,10.25) .. controls (4,10) and (4,10) .. (4,9.75);
\draw [line width=0.6pt, short] (16.75,10.25) -- (16.75,9.75);
\draw [line width=0.6pt, short] (19.5,10.25) -- (19.5,9.75);
\draw [line width=0.6pt, short] (18,10.25) -- (18,9.75);
\draw [line width=0.6pt, short] (5.25,10.25) -- (5.25,9.5);
\draw [line width=0.6pt, short] (3.25,10.25) -- (3.25,9.75);
\node [font=\LARGE] at (13,9.25) {-$\pi$/4};
\node [font=\LARGE] at (13,8.25) {-$\pi$/2};
\node [font=\LARGE] at (14,14.5) {$\phi$};
\node [font=\LARGE] at (14,10.75) {0};
\node [font=\Large] at (14.75,9.75) {1};
\node [font=\LARGE] at (15.5,9.5) {10};
\node [font=\LARGE] at (16.75,9.25) {$10^2$};
\node [font=\LARGE] at (20,9.5) {$\omega$};
\node [font=\LARGE] at (18,9.25) {$10^3$};
\node [font=\LARGE] at (1.5,9.25) {-10};
\node [font=\LARGE] at (-1.75,14.25) {$20log(| \frac{V_0(\omega)}{V_i(\omega)} |)$};
\node [font=\LARGE] at (5.75,9) {$10^3$};
\node [font=\LARGE] at (3.25,9.25) {10};
\node [font=\LARGE] at (4.25,9) {$10^2$};
\node [font=\LARGE] at (1.5,7.25) {-30};
\node [font=\LARGE] at (1.5,8.25) {-20};
\node [font=\LARGE] at (2.75,10.5) {1};
\node [font=\LARGE] at (2,10.5) {0};
\node [font=\LARGE] at (-0.5,6.5) {$10^3$};
\node [font=\LARGE] at (8,9.25) {$\omega$};
\node [font=\LARGE] at (13.5,13.5) {$\pi$/2};
\node [font=\LARGE] at (13.25,11.75) {$\pi$/4};
\draw [line width=0.6pt, short] (14.25,11.75) -- (14.75,11.75);
\draw [line width=0.6pt, short] (14.25,13) -- (14.75,13);
\end{circuitikz}
}%


\end{document}

\end{enumerate}
\item An output device is interfaced with 8-bit microprocessor 8085A. The interfacing circuit is shown in
figure 


\begin{center}
\documentclass{standalone}
\usepackage{amsmath, tikz}
\usepackage{circuitikz}
\usepackage{amsmath}
\begin{document}

\centering
\resizebox{0.65\textwidth}{!}{%
\begin{circuitikz}
\tikzstyle{every node}=[font=\LARGE]



\draw [ line width=0.8pt ] (14.75,23.75) rectangle (21.5,13);
\draw [ line width=0.8pt ] (25,23) rectangle (32.5,19.75);
\draw [ line width=0.8pt ] (26.25,17) rectangle (31.25,15.5);
\node [font=\LARGE] at (34.5,27.5) {AD};
\node [font=\LARGE] at (35,25.5) {BDB};
\node [font=\huge] at (28.5,21.5) {Output Port};
\node [font=\LARGE] at (29,15) {Output Device};
\node [font=\huge] at (17.5,24.5) {3Lx8L Decoder};
\node [font=\LARGE] at (15.5,23) {$I_2$};
\node [font=\LARGE] at (15.5,22) {$I_1$};
\node [font=\LARGE] at (15.5,20.5) {$I_0$};
\node [font=\LARGE] at (15.5,17.5) {$E_1$};
\node [font=\LARGE] at (15.5,15) {$\overline{E_2}$};
\node [font=\LARGE] at (21,22.25) {1};
\node [font=\LARGE] at (21,21) {2};
\node [font=\LARGE] at (21,19.75) {3};
\node [font=\LARGE] at (21,18.5) {4};
\node [font=\LARGE] at (21,17.25) {5};
\node [font=\LARGE] at (21,15.75) {6};
\draw [ line width=1.2pt](8.25,27.5) to[short] (8.25,17.25);
\node at (21.75,23.25) [ocirc, scale= 4] {};
\node at (21.75,22.25) [ocirc, scale= 4] {};
\node at (21.75,19.75) [ocirc, scale= 4] {};
\node at (21.75,18.5) [ocirc, scale= 4] {};
\node at (21.75,15.75) [ocirc, scale= 4] {};
\node at (21.75,17.25) [ocirc, scale= 4] {};
\node at (21.75,21) [ocirc, scale= 4] {};
\draw [ line width=1.4pt](28.75,19.75) to[multiwire] (28.75,17.25);
\draw [line width=1.2pt, short] (28.75,17.25) -- (28.5,17.75);
\draw [line width=1.2pt, short] (28.75,17.25) -- (29,17.75);
\draw [ line width=1pt](7.5,25.5) to[short] (29.5,25.5);
\draw [ line width=1.3pt](31,11.25) to[multiwire] (34.25,11.25);
\draw [ line width=1.3pt](8,11.25) to[short] (31,11.25);
\draw [ line width=1.4pt](28.75,25.5) to[multiwire] (28.75,23);
\draw [line width=1.2pt, short] (28.75,23) -- (28.5,23.5);
\draw [line width=1.2pt, short] (28.75,23) -- (29,23.5);
\draw [ line width=1.5pt](7.75,27.5) to[short] (34,27.5);
\node [font=\LARGE] at (21,14.25) {7};
\node at (21.75,14.25) [ocirc, scale= 4] {};
\node [font=\LARGE] at (21,23.25) {0};
\node [font=\LARGE] at (7.25,18.75) {$A_{11}$};
\node [font=\LARGE] at (10.25,19) {$A_{12}$};
\node [font=\LARGE] at (10.5,20.5) {$A_{13}$};
\node [font=\LARGE] at (12,22.5) {$A_{14}$};
\node [font=\LARGE] at (13,23.75) {$A_{15}$};
\node [font=\LARGE] at (6,12.25) {IO/ $\overline{M}$};
\draw [ line width=1pt](22,21) to[short] (23,21);
\draw [ line width=1pt](23,21) to[short] (23,22);
\draw [ line width=1pt](23,22) to[short] (24.5,22);
\node at (24.75,22) [ocirc, scale= 4] {};
\draw [ line width=1pt](11.25,13.5) to[short] (11.25,11.25);
\draw [ line width=1pt](11.25,13.5) to[short] (14.25,13.5);
\node at (14.5,13.5) [ocirc, scale= 4] {};
\node at (14.5,14.75) [ocirc, scale= 4] {};
\draw [ line width=1pt](12.25,27.5) to[short] (12.25,23);
\draw [ line width=1pt](12.25,23) to[short] (14.75,23);
\draw [ line width=1pt](11.25,27.5) to[short] (11.25,22);
\draw [ line width=1pt](11.25,22) to[short] (14.75,22);
\draw [ line width=1pt](10.5,27.5) to[short] (10.5,21);
\draw [ line width=1pt](10.5,21) to[short] (14.75,21);
\draw [ line width=1pt](8.5,11.25) to[short] (8.5,14.75);
\draw [ line width=1pt](9.5,27.5) to[short] (9.5,18.25);
\draw [ line width=1pt](8.5,14.75) to[short] (14.25,14.75);
\node [font=\LARGE] at (15.5,13.75) {$\overline{E_3}$};
\node [font=\LARGE] at (30.25,26.25) {8};
\node [font=\LARGE] at (14,12.25) {$\overline{WR}$};
\node [font=\LARGE] at (30,24.5) {8};
\node [font=\LARGE] at (29.5,18.5) {8};
\node [font=\LARGE] at (35,11.25) {BCB};
\draw [line width=1.3pt, short] (34,27.5) -- (33,27.75);
\draw [line width=1.2pt, short] (34,27.5) -- (33,27.25);
\draw [line width=1.7pt, short] (7.5,25.5) -- (30.5,25.5);
\draw [line width=1.3pt, short] (34.25,11.25) -- (33.5,11.5);
\draw [line width=1.2pt, short] (34.25,11.25) -- (33.5,11);
\draw [line width=1.3pt, short] (31.25,25.25) -- (30.5,25.5);
\draw [line width=1.2pt, short] (31.25,25.75) -- (30.5,25.5);
\draw [line width=1.3pt, short] (34,25.5) -- (33,25.75);
\draw [line width=1.2pt, short] (34,25.5) -- (33.25,25.25);
\draw [line width=1.3pt, short] (31.5,11) -- (30.75,11.25);
\draw [line width=1.2pt, short] (31.5,11.5) -- (30.75,11.25);
\draw [ line width=1.3pt](30.25,25.5) to[multiwire] (34,25.5);

\draw (10.75,18.25) node[ieeestd and port, anchor=in 1, scale=1.8](port){} (port.out) to[short] (14.75,17.75);
\node at (10.5,17.25) [ocirc, scale= 4] {};
\draw [ line width=1.3pt](9.5,18.25) to[short] (10.75,18.25);
\draw [ line width=1.3pt](8.25,17.25) to[short] (10.25,17.25) ;
\end{circuitikz}
}



\end{document}

\end{center}


The interfacing circuit makes use of 3 Line to 8 Line decoder having 3 enable lines $E_1$, $\overline{E_2}$, $\overline{E_3}$. The
address of the device is\hfill{(2014-EE)} 

	\begin{enumerate}
		    \begin{multicols}{4} 
	\item $50_H$
	\item   $5000_H$
	\item  $A0_H$
	\item   $A000_H$
\end{multicols}  

\end{enumerate}

\item The figure shows the circuit diagram of a rectifier. The load consists of a resistance 10 $\ohm$ and an
inductance 0.05 $H$ connected in series. Assuming ideal thyristor and ideal diode, the thyristor
firing angle (in degree) needed to obtain an average load voltage of 70 $V$ is \underline{\hspace{2.5 cm}}  \hfill{(2014-EE)}

\begin{center}
\documentclass{standalone}
\usepackage{amsmath, tikz}
\usepackage{circuitikz}

\begin{document}

\centering
\resizebox{0.5\textwidth}{!}{%
\begin{circuitikz}
\tikzstyle{every node}=[font=\normalsize]
\draw (3.75,14.75) to[sinusoidal voltage source, sources/symbol/rotate=auto] (3.75,10.5);
\draw (3.75,10.5) to[short] (12.25,10.5);
\draw (12.25,14.75) to[european resistor] (12.25,10.5);
\draw (10.25,10.5) to[D] (10.25,14.75);
\node [font=\LARGE] at (2,13.75) {};
\node [font=\large] at (13,12.75) {load};
\node [font=\large] at (3.25,12) {+};
\node [font=\normalsize] at (8.5,14.35) {$\mid$};
\node [font=\large] at (1.75,12.75) {$325sin(314t)V$};
\node [font=\Large] at (3.25,13.25) {-};

\draw (3.75,14.75) to[D] (12.25,14.75);
\draw [ line width=0.75pt]  (8.26,14.651) to[short] (8.5,14.5);
\end{circuitikz}
}%



\end{document}

\end{center}




\item Figure (i) shows the circuit diagram of a chopper. The switch S in circuit in figure (i) is switched such that the voltage $v_D$ across the diode has the wave shape as shown in figure (ii). The capacitance $C$ is large so that the voltage across it is constant. If switch S and the diode are ideal, the peak to peak ripple(in A) in the inductor current is  \underline{\hspace{2.5 cm}}    \hfill{(2014-EE)} 
         
\begin{center}
\documentclass{standalone}
\usepackage{amsmath, tikz}
\usepackage{circuitikz}

\begin{document}
\centering
\resizebox{0.6\textwidth}{!}{%
\begin{circuitikz}
\tikzstyle{every node}=[font=\LARGE]
\draw [ line width=0.7pt](24.5,18.5) to[european resistor] (24.5,14.5);
\draw [line width=0.7pt](21.25,18.5) to[C] (21.25,14.5);
\draw [ line width=0.8pt](21.25,14.5) to[short] (24.5,14.5);
\draw (10,18.5) to[battery1] (10,14.5);
\draw [ line width=0.9pt](15.25,14.5) to[D] (15.25,18.5);
\draw [ line width=0.7pt](21.25,14.5) to[short] (10,14.5);
\draw [ line width=0.7pt](17.25,18.5) to[short] (14.5,18.5);
\draw [line width=0.7pt, short] (17.25,18.5) .. controls (17.5,18.75) and (17.75,18.75) .. (18,18.5);
\draw [line width=0.7pt, short] (19.5,18.5) -- (24.5,18.5);
\draw [line width=0.7pt, short] (18,18.5) .. controls (18.25,18.75) and (18.5,18.75) .. (18.75,18.5);
\draw [line width=0.7pt, short] (18.75,18.5) .. controls (19,18.75) and (19.25,18.75) .. (19.5,18.5);
\node [font=\normalsize] at (18.5,19) {1mH};
\node [font=\large] at (22,16.75) {C};
\node [font=\large] at (25.5,16.5) {Load};
\node [font=\large] at (14.5,16.5) {$V_D$};
\node [font=\Large] at (14.5,18) {+};
\node [font=\LARGE] at (14.5,15.5) {-};
\node [font=\large] at (9,16.5) {100V};
\draw [ line width=0.7pt](10,18.5) to[short, -o] (11.75,18.5) ;
\draw [ line width=0.7pt](14.5,18.5) to[short, -o] (12.5,18.5) ;
\draw [line width=0.8pt, short] (11.75,19) -- (12.5,18.5);
\node [font=\large] at (12.5,19) {S};
\end{circuitikz}
}%



\end{document}

Figure(i)
\end{center}

\begin{center}
\documentclass{standalone}
\usepackage{amsmath, tikz}
\usepackage{circuitikz}

\begin{document}

\centering
\resizebox{0.6\textwidth}{!}{%
\begin{circuitikz}
\tikzstyle{every node}=[font=\Large]
\draw [->, >=Stealth,line width=1.5pt] (6,19) -- (6,26.5);
\draw [->, >=Stealth,line width=1.5pt] (6,19) -- (26.25,19);
\draw [line width=1.5pt, short] (6,19.25) -- (6,24);
\draw [line width=3pt, short] (6,24) -- (10.5,24);
\draw [line width=3pt, short] (10.5,24) -- (10.5,19);
\draw [line width=3pt, short] (6,24) -- (6,19);
\draw [line width=3pt, short] (24,24) -- (25.75,24);
\draw [line width=3pt, short] (19.5,19) -- (24,19);
\draw [line width=3pt, short] (10.5,19) -- (15,19);
\draw [line width=3pt, short] (15,24) -- (19.5,24);
\draw [line width=3pt, short] (24,24) -- (24,19);
\draw [line width=3pt, short] (19.5,24) -- (19.5,19);
\draw [line width=3pt, short] (15,24) -- (15,19);
\draw (6,19) to[line width=1.5pt ] (6,18.25);
\node [font=\Large] at (6.25,18.5) {0};
\node [font=\Large] at (10.5,18.5) {0.05};
\node [font=\Large] at (15,18.5) {0.1};
\node [font=\Large] at (24,18.5) {0.2};
\node [font=\Large] at (25.5,18.5) {t(m/s)};
\node [font=\large] at (5,26.25) {$V_D$};
\node [font=\Large] at (19.5,18.5) {0.15};
\node [font=\Large] at (5.25,24.25) {$100V $};
\end{circuitikz}
}%
\label{Figure(ii)}




\end{document}

Figure(ii)
\end{center}


\item The figure shows one period of the output voltage of an inverter. $\alpha$ should be chosen such that $60 ^{\circ} < \alpha<90^{\circ}$. If the rms value of fundamental component is 50 V, then $\alpha$ in degree  is \underline{\hspace{2.5 cm}}  \hfill{(2014-EE)}

\begin{center}
\documentclass{standalone}
\usepackage{amsmath, tikz}
\usepackage{circuitikz}

\begin{document}
\centering
\resizebox{1\textwidth}{!}{%
\begin{circuitikz}
\tikzstyle{every node}=[font=\LARGE]

\draw [short] (1.75,2.25) -- (1.75,8.75);
\draw [short] (-4,2.5) -- (-4,8.75);
\draw [short] (3.25,2.25) .. controls (3.25,5) and (3.25,4.75) .. (3.25,8.75);
\draw [short] (-4,8.75) -- (1.75,8.75);
\draw [short] (1.75,2.25) -- (3.25,2.25);
\draw [short] (9,2.25) -- (9,8.75);
\draw [short] (3.25,3.25) -- (3.25,8.75);
\draw [short] (22,2.25) .. controls (22,5.75) and (22,4.75) .. (22,8.75);
\draw [short] (3.25,8.75) -- (9,8.75);
\node [font=\LARGE] at (-0.5,8.25) {};
\node [font=\LARGE] at (-0.5,8.25) {};
\node [font=\LARGE] at (6.5,9.25) {100V};
\node [font=\LARGE] at (2.25,1.5) {-100V};
\node [font=\LARGE] at (-1,9.25) {100V};
\node [font=\LARGE] at (19,1.5) {-100V};
\node [font=\LARGE] at (15.5,9) {100V};
\node [font=\LARGE] at (1.25,5.25) {$\alpha$};
\node [font=\LARGE] at (-3.75,5.25) {0};
\node [font=\LARGE] at (11.5,1.5) {-100V};
\node [font=\LARGE] at (8,5.25) {180};
\node [font=\LARGE] at (4.25,5) {180-$\alpha$};
\node [font=\LARGE] at (21.25,5.25) {360};
\node [font=\LARGE] at (17.25,5.25) {360-$\alpha$};
\node [font=\LARGE] at (23.5,5.25) {$\omega t$};
\node [font=\LARGE] at (13.5,5.25) {180+$\alpha$};
\node [font=\LARGE] at (23.25,4.5) {(degree)};
\draw [->, >=Stealth] (-4.5,5.75) -- (24.5,5.75);
\draw [short] (9,2.25) -- (14.75,2.25);
\draw [short] (16.25,2.25) -- (22,2.25);
\draw [short] (14.75,8.75) -- (14.75,2.25);
\draw [short] (16.25,8.75) -- (16.25,2.25);
\draw [short] (14.75,8.75) -- (16.25,8.75);
\end{circuitikz}
}

\end{document}

\end{center}
\begin{center}
    \textbf{END OF THE QUESTION PAPER}
    \end{center}
\end{enumerate}


\end{document}
